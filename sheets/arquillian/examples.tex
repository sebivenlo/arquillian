\section[FX Properties]{JavaFX Properties and Binding}
\begin{frame}{FX Beans vs Java Beans}
  \begin{itemize}
  \item more direct binding/ relationship of and between variables.
    \begin{itemize}
    \item The work around triggering and binding does no longer take
      place in containing class.
    \item because the properties are ``exposed'', so can manipulated from the outside.
    \end{itemize}

  \item new (extra) convention for getter of property. \Code{xXXXProperty()}
    where xXXX is the name of the property.
  \end{itemize}
\end{frame}

\begin{frame}{Meet Bill}
\lstset{basicstyle={\scriptsize},upquote=true,language=Java}
  \lstinputlisting[firstline=6,title={Simple bill example}]{../code/fxbindingdemo/src/demo1/Bill.java}
\end{frame}

\begin{frame}{Improvements}
  \begin{itemize}
  \item JavaFX properties expands and improves the JavaBeans model.
  \item \Okis{Binding} is assembled from one or more sources, the
    \textit{dependencies}.
  \item A binding observes its dependencies for changes and updates itself
    after change detection, but lazily.
  \item Lazy means, that the value is recomputed on demand (when getter
    is called) in stead of every time it is set.
  \item The documentation also states that the implementer should
    minimize the work in event handlers: \textit{\enquote{Implementations of this class should strive to generate as few events as possible to avoid wasting too much time in event handlers.}}
  \item So:
    \begin{itemize}
    \item if the new value is the same as the previous: do not
      bother.
  \item If you are already invalid, do not propagate this news any further.
  \end{itemize}
  
\end{itemize}
\end{frame}

\begin{frame}{Property APIs}
  JavaFX beans provide three way to deal with bound properties:
  \begin{itemize}
  \item High level API.
    \begin{itemize}
    \item Fluent API
    \item Bindings utility class.
    \end{itemize}
  \item Low level API.
  \end{itemize}
\end{frame}

\begin{frame}{High level API, Fluent}
  Make use of the (computational) method in the property:
  \Code{NumberProperty} and its children.

  \lstinputlisting[basicstyle={\tiny},firstline=9,lastline=17,title={Fluent
  API}]{../code/fxbindingdemo/src/demo1/FluentDemo.java}

Note the use of \textit{chaining} the method calls. It is because
the \Code{add} etc. methods return an object which support the
operation.

This is called a \textit{fluent} style of programming.

\end{frame}

\begin{frame}{High level API, Bindings class}

  The same computation can be done using the \Code{Bindings} class
  which provides \textit{factories} for all kinds of binding.

  \lstinputlisting[basicstyle={\tiny},firstline=10,lastline=19,title={Using
  Bindings},morekeywords={add,multiply}]{../code/fxbindingdemo/src/demo1/UsingBindings.java}

The \Code{add} and \Code{multiply} methods stem from the Bindings class.

  \lstinputlisting[basicstyle={\tiny},firstline=3,lastline=3,title={The import statement}]{../code/fxbindingdemo/src/demo1/UsingBindings.java}

modified from javafx bindings tutorial.
\end{frame}

\begin{frame}%{Fluent API and Bindings combined}

  \lstinputlisting[basicstyle={\tiny},firstline=10,lastline=20
  ,title={Combining Fluent API and Bindings}]%
  {../code/fxbindingdemo/src/demo1/Combining.java}
  \begin{itemize}
  \item Convenient to transform a formula to binding.
  \item When mixing types (e.g. int, double), conversion (promotion)
    can take place. The rules are the same as in Java.
  \end{itemize}
  {\scriptsize
    \begin{enumerate}
    \item If one of the operands is a double, the result is a double.
    \item If not and one of the operands is a float, the result is a float.
    \item If not and one of the operands is a long, the result is a long.
    \item The result is an integer otherwise.
    \end{enumerate}
    from javafx bindings tutorial.
  }
\end{frame}

\begin{frame}{Lazy evaluation}

  \lstinputlisting[basicstyle={\tiny},firstline=14,lastline=33
  ,title={Lazy evaluation in action}]%
  {../code/fxbindingdemo/src/demo1/UsingInvalidation.java}


  The title is probably an oxymoron.
\end{frame}

\begin{frame}
  \lstinputlisting[basicstyle={\tiny},firstline=11,lastline=30
  ,title={Low Level API}]%
  {../code/fxbindingdemo/src/demo1/UsingLowLevelApi.java}

{\scriptsize
  \begin{itemize}
  \item The low level API can give more control and performance (
    avoids intermediate objects)  in return for a little more work
  \item Note the use of the block, which effectively is the sole constructor.
  \item \Code{super} of DoubleBinding takes care of invalidation behavior. 
  \end{itemize}
}


\end{frame}
